\chapter{绪论}

\section{研究背景及意义}
听觉脑干反应是一种客观评估听觉通路功能的重要方法,其原理是通过对外耳施加短促声刺激(如Click或 Chirp 刺激),记录从耳蜗至脑干一系列神经元放电所产生的微电位变化。ABR 波形主要包括七个峰值(I–VII波),其中最常用于诊断的是前五个波,特别是 I、III 和 V 波。V 波因其振幅大、信噪比高,在实际应用中常作为阈值判定依据。传统的 ABR 波形判读依赖听力专家根据波形形状、潜伏期等参数进行人工标注,这种方式不仅效率低、主观性强,而且在存在噪声干扰或病理性波形时极易造成误判。传统的 ABR 检查通常需要重复多次平均以提高信噪比(SNR),在特定频率和刺激强度下,单个耳朵的检测可能耗时达 30 分钟以上。对新生儿、老年人或注意力难以集中的人群来说,长时间保持安静状态极具挑战性,这直接影响了测试结果的准确性与筛查效率。

因此,实现 ABR 数据的快速采集不仅能够显著减少患者在测试过程中的等待与不适时间,提升临床操作的效率;同时,借助人工智能技术对 ABR 波形进行自动化与智能化分析,能够降低人工判读的主观性,提高诊断的准确率和一致性。这两方面的协同发展,构成了当前听力诊断技术研究的重要方向。

过去数十年来,研究者们尝试了多种自动化ABR分析方法,主要包括:

\textbf{基于波形相似性的方法:}
其核心思想是通过比较记录的ABR波形与已知模板或训练数据的相似性来判断反应是否存在。其中包括两大类,一是模板匹配\cite{valderrama2014} 即预先定义标准波形,计算测试波形与模板的相关系数或均方误差。但由于个体间波形形态差异大,模板难以通用。二是人工神经网络,使用机器学习ABR特征,自动分类“有反应”或“无反应”。其特点是需要依赖大量训练数据,且不同设备/实验室的数据分布可能不同,基于医学数据的采集受限和患者隐私数据保护,泛化能力受限。
\section{研究对象}

\section{研究方法}

\chapter{数学基础}

\section{基础公设}