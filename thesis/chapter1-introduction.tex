\chapter{绪论}

\section{研究背景及意义}
听觉脑干反应做为一种客观评估听觉通路功能的重要方法,其原理是通过对外耳施加短促声刺激(如Click或 Chirp 刺激),记录从耳蜗至脑干一系列神经元放电所产生的微电位变化。ABR 波形主要包括七个峰值(I–VII 波),其中最常用于诊断的是前五个波,特别是 I、III 和 V 波。V 波因其振幅大、信噪比高,在实际应用中常作为阈值判定依据。传统的 ABR 波形判读依赖听力专家根据波形形状、潜伏期等参数进行人工标注,这种方式不仅效率低、主观性强,而且在存在噪声干扰或病理性波形时极易造成误判。传统的 ABR 检查通常需要重复多次平均以提高信噪比,在特定频率和刺激强度下,单个耳朵的检测可能耗时达 30 分钟以上。对新生儿、老年人或注意力难以集中的人群来说,长时间保持安静状态极具挑战性,这直接影响了测试结果的准确性与筛查效率。

因此,实现 ABR 数据的快速采集不仅能够显著减少患者在测试过程中的等待与不适时间,提升临床操作的效率;同时,借助人工智能技术对 ABR 波形进行自动化与智能化分析,能够降低人工判读的主观性,提高诊断的准确率和一致性。这两方面的协同发展,构成了当前听力诊断技术研究的重要方向。

过去数十年来,研究者们尝试了多种自动化ABR分析方法,主要包括:

\textbf{基于波形相似性的方法:}
其核心思想是通过比较记录的 ABR 波形与已知模板或训练数据的相似性来判断反应是否存在。其中包括两大类,一是模板匹配\cite{valderrama2014} 即预先定义标准波形,计算测试波形与模板的相关系数或均方误差。但由于个体间波形形态差异大,模板难以通用。二是人工神经网络,使用机器学习 ABR 特征,自动分类“有反应”或“无反应”。其特点是需要依赖大量训练数据,且不同设备/实验室的数据分布可能不同,基于医学数据的采集受限和患者隐私数据保护,泛化能力受限。

\textbf{基于波形稳定性的方法:}
利用其神经活动的锁相特性,通过分析多次扫描之间的一致性来判断反应的真实性。其中一种常见方法是单次扫描互相关分析,即计算相邻 ABR 扫描之间的相关系数,高相关性表明神经反应具有良好的重复性和稳定性。

多项研究表明ABR对 Click 或 Chirp 刺激的极性敏感性显著影响其波形表现。例如,de Lima 等人(2008)\cite{delima2008polarity} 研究发现在稀疏、压缩和交替三种极性条件下,稀疏极性产生了波 I、III 和 V 的最短潜伏期和最高幅度信号,此外,一项结合 Click 刺激速率与声压级的研究发现,在常见重复率下两种极性对波形的影响有限,但在低重复率或特定声级条件下仍有差异表现。最近,Dzulkarnain 等人(2021)\cite{dzulkarnain2021influence}首次在 LS‑Chirp 刺激条件下比较三种极性,其结果也显示稀疏极性能显著提升波形振幅与信噪比,且潜伏期变化极小,因而被推荐用于临床应用中以提升 ABR 响应质量。

在刺激强度变化趋势分析方面,Suthakar 与 Liberman\cite{Suthakar_Liberman2019} 提出了一种跨强度水平的互相关分析法,通过分析不同刺激强度下 ABR 波形之间的相似性趋势来判断神经响应是否真实存在。该方法无需依赖主观判断,且与临床专家的视觉识别结果高度一致,为实现自动化阈值检测流程提供了有力支持。

\textbf{基于信号质量的评分方法:}
近年来,统计学检测方法如 F$_{sp}$ 比值、FMP 和 Hotelling's T² 检验在 ABR 自动判定中得到广泛研究。其中,Chesnaye 等人(2018)通过仿真与真实数据对比,发现 Hotelling's T² 检验在灵敏度与检测时间兼具优势,且结合 Bootstrap 方法的 F$_{sp}$/FMP 显著提高误报控制能力。逐步判决测试的 Hotelling's T² 方法被认定具有很高的临床可行性。同时,Katlin 等人(2025)\cite{Katlin2025}的研究表明,在 60 dB nHL 等较低刺激条件下,当 F$_{sp}$ 值超过阈值时仍能可靠识别出 ABR 响应,提示 F$_{sp}$ 可作为快速筛查辅助判定工具。但所有这些方法在低刺激强度和瞬态波检测方面仍存在局限,因此仍需结合模型算法或信号增强方法共同优化阈值检测系统。

\textbf{基于生理建模的评分方法:}
这类方法通过建模 ABR 的生理特性,如强度-潜伏期函数关系来客观评估反应,减少对固定模板的依赖。典型代表是强度-潜伏期拟合技术(Schilling et al., 2019\cite{Schilling2019}),其通过分析不同刺激强度下V波潜伏期的变化规律(通常表现为强度降低时潜伏期延长),建立数学模型来外推听觉阈值。这种方法利用了听觉通路的固有生理特征,避免了模板匹配中个体波形差异带来的问题。然而,其应用存在明显限制:一方面,该方法主要适用于瞬态刺激(如 Click-ABR ),对频率特异性刺激(如 Tone-Burst ABR )的适应性较差;另一方面,个体神经传导速度的差异可能导致模型预测偏差,影响阈值检测的准确性。因此,这类方法更适合作为新生儿听力筛查中的Click-ABR特定场景下的辅助分析工具。

然而,由于受试者个体差异、电极放置/阻抗变化以及采集设置等因素导致的波形异质性和信噪比变化,现有方法只能在有限的实验条件下实现准确的阈值测定,这严重限制了不同实验室间ABR结果的可比性。使用什么方面获得最优的刺激和强度至今还没有特别合适的方式。


尽管 ABR 检测本身是客观的测量方法,但阈值的确定仍依赖于人工对波形的判读。这一过程需要训练有素的技术人员监督波形识别,不仅耗时耗力,而且由于个人技能和经验差异,特别是面对非典型波形或高背景噪声时,容易引入主观误差。

\textbf{ABR 阈值的确定方法} 主要包括人工判读法、统计学检测法、临界声压法、波形趋势外推法以及近年来兴起的基于人工智能的自动检测算法等。
\begin{itemize}
  \item \textbf{人工判读法}:人工判读法是最传统也是最常见的方式,通常由具有经验的临床人员观察波形中是否存在稳定且重复的波V成分来判断反应的存在。该方法对操作者专业水平要求较高,主观性较强,但在复杂或边界模糊的病例中具有较强的灵活性。

  \item \textbf{统计学检测法}:相较之下,统计学检测法则通过算法(如相干分析、t检验、Fisher判定等)客观判断ABR波形的显著性。例如有的系统采用置信区间估计来判定波形的存在,从而减少人为误差,提高检测的一致性与效率,尤其适合新生儿听力筛查等自动化要求高的场景。

  \item \textbf{临界声压法}:临界声压法则结合人工和客观观察,以较高声压级开始施测,逐步下降,每次递减5-10 dB,直到波形消失,再略微上调确认波V的再现性,以最终界定阈值。这种方法简单直观,临床适用性强。对于高精度要求的研究或复杂病例,还可通过分析不同声压下波V的幅度和潜伏期变化趋势,采用外推技术预测最小有效刺激水平,这种方法虽然计算量较大,但能提供更连续和量化的听力估计。

  \item \textbf{基于深度学习的方法}:近年来,随着机器学习与人工智能技术的发展,越来越多的研究开始探索基于深度学习模型对 ABR 波形进行自动识别和阈值估计。这类方法可以在海量数据中学习特征,提高检测灵敏度,并在自动化听力筛查设备中初步得到应用。
\end{itemize}

\section{研究内容}
为了优化ABR测试流程并缩短检测时间,本文首先从刺激信号设计、刺激速率提升以及信号处理降噪三大方向展开了深入探索。在刺激信号方面,设计更符合耳蜗生理响应特性的信号成为关键研究重点。例如,啁啾(Chirp)信号被广泛采用用于补偿基底膜传播的时间延迟,从而提升听神经的同步性并增强ABR波形幅度。这一思路虽然有效,但仍面临着如何针对不同听力状态个体化设定调频参数的技术难题。
与此同时,提升刺激速率也被视为缩短测试时间的有效途径。例如,通过使用交错频率刺激或最大长度序列(MLS)等策略,系统能够在单位时间内获取更多响应数据。然而,高速率刺激往往伴随着反应适应性增强及波形重叠的问题,因此对响应信号的解卷积和复原算法提出了更高要求。
在信号处理方面,为了提高测试的稳定性与准确性,研究者尝试引入自适应滤波、多通道降噪与时频分析等方法来降低背景噪声的干扰。这些手段虽在理论上具有显著优势,但其在实际应用中需平衡算法复杂度与运算效率,且对不同类型噪声的鲁棒性仍需系统验证。


在本研究中,本文系统地评估了以上三类ABR测试优化策略在实际应用中的有效性与适用性。研究不仅比较了不同刺激信号类型(如宽带或窄带Chirp、MLS、交错频率刺激)在多种频率和声强下的诱发效果,还深入分析了各类降噪技术对ABR波形质量的影响。本文同时引入F$_{sp}$比值与Hotelling's T²统计量作为客观评估指标,建立了一套可量化的ABR质量分析框架,并在此基础上提出了针对不同临床听力筛查场景下的最优刺激与信号处理组合建议,为后续自动化诊断系统的开发奠定了基础。


本研究的第二部分中,本文将重点放在了基于深度学习的 ABR波形自动分析上,目标是探索如何利用端到端的神经网络模型对 ABR 波形进行有效识别与分类,从而辅助实现听力筛查自动化与智能化。
针对传统 ABR 分析方法存在依赖手动特征提取、抗噪能力差的问题,本文决定研究使用一种基于 CNN-BiLSTM 组合结构的深度学习模型。该模型结合了卷积神经网络对原始 ABR 信号的局部时序特征提取能力,以及BiLSTM对信号整体时序依赖建模的优势,能够从整体上学习 ABR 波形在不同时间片段的变化趋势。

\section{本文主要贡献}
首先,本文系统评估并比较了多种刺激信号类型及信号处理方法在不同频率和声压级条件下对ABR波形检测性能的影响。研究发现,最优的Wave V振幅与F$_{sp}$比值因刺激频率段(宽带 vs 窄带)及呈现声级而异。例如,在75 dB SL下,Click或纯音刺激配合常规速率与自适应滤波组合表现最佳;而在40 dBSL下,宽带 Chirp 信号或 MLS 刺激在一定条件下取得更高的神经同步性和响应幅度。对于低频(500 Hz)刺激,使用交替极性的音调信号并结合Wiener滤波处理,能够更有效地检测到Wave V反应。上述结果为不同听力水平下的ABR刺激参数优化提供了实证依据。

其次,本文重点验证了Wiener滤波为代表的自适应滤波方法在低信噪比ABR数据处理中展现出的稳定性与广泛适用性。无论在500 Hz、4 kHz或宽带刺激条件下,采用该类滤波算法均显著提升了F$_{sp}$比值并增强了Wave V识别率。特别是在传统处理方法难以识别微弱波形或响应尚不明确的情形下,自适应滤波展现出良好的抑噪能力和临床适用潜力,证明其可作为ABR常规处理流程中的关键环节之一。

再次,基于上述刺激与处理参数优化实验的成果,本文进一步引入多种深度学习方法,对ABR波形进行自动识别与分析,重点比较了卷积神经网络、SVM、融合的 CNN+BiLSTM 混合架构在实际数据集上的表现。实验结果显示,CNN+BiLSTM模型在准确率、稳定性以及对弱波形的识别能力方面均优于单一模型结构,尤其在复杂背景噪声条件下仍能保持较高的检测性能。该结果充分证明,在已有波形增强与优化处理的基础上,端到端的深度学习模型可进一步提升 ABR 自动分析的智能化水平。


最后,本文提出了一种面向临床应用的ABR优化框架,将刺激参数选择、信号预处理方法、模型结构设计三者有机整合,并通过实测数据的对比分析建立了一套完整的“刺激–处理–识别”工作链。在数据方面,本文基于真实ABR测量流程,自行采集并标注多组样本数据,覆盖多个频段与声级条件,弥补了公开数据集缺乏的局限性。研究还指出,在常规2000–3000 sweeps测量时间内,若刺激强度过低(如30–40 dBSL),可能难以获得稳定反应,提示后续需探索更高效的刺激编码方案或数据增强策略,以进一步提升检测灵敏度与速度。

\section{本文章节安排}
本文正文总共分为五个章节,分别为绪论,相关工作,基于多通道多声强的ABR刺激增强方法,基于深度学习的ABR识别方法和总结与展望。以下是每个章节的详细内容:

1. 绪论,主要介绍ABR的研究背景和意义,回顾了其发展历程,并总结了当前主流的四类自动化分析技术及其关键方法。针对现有方法在泛化能力和诊断效率方面的不足,本文提出了结合刺激优化与深度学习的新型分析方案,并概述了本文的研究内容与主要贡献。

2. 相关工作,本节对本文所处的 ABR 快速诊断领域进行了系统性综述,首先对 ABR 的支撑技术进行了简要概述,为后文方法部分提供必要的背景知识;其次,从刺激参数设计、刺激类型与强度调控等多个方面,对 ABR 快速诊断的关键环节进行了系统阐述;此外,还对机器学习与深度学习在 ABR 波形分析中的应用研究进行了综述,为后续第三、四章方法部分的提出提供了完整的技术背景和理论支撑。

3. 增强方法,主要介绍本文提出的基于多通道多声强的 ABR 刺激增强方法,详细阐述方法的设计思路与整体框架。首先介绍刺激策略的总体设计理念;其次,从多刺激组合、声强动态调节两方面,刺激以75 dB为高强度,40 dB为低强度,构建多种类多声强配合模型;然后描述信号优化模块的实现,最后,通过实验数据对响应判别与输出评估流程进行验证,展示本方法在提高 ABR 信噪比与检测效率方面的优势。

4. 识别方法,根据第三章的的优化刺激类型和信号处理方法,获得的ABR数据,以信号的分类和关键V波峰的识别做为主要研究目标,阐述了创新点以及代表性的深度学习的方法,设计了 CNN+BiLSTM 的混合模型架构,并做了相关的实验,与其它模型的实验结果做对比,得出该模型的优势。

5. 总结与展望,主要对全文内容进行总结,同时指出工作中遇到的问题,总结后续的工作方向并对该领域未来发展做出展望。