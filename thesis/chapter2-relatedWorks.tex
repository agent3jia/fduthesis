\chapter{相关工作}

\section{技术支撑}
ABR的核心在于精准选择最优的ABR刺激声类型, 并确定最佳刺激强度,其核心基础在于刺激声参数的优化选择。这直接决定了神经反应的敏感性和特异性。科学的刺激参数设置能显著提升信号质量,为后续分析提供可靠基础。本节前半部分将从刺激声类型、刺激强度和刺激速率三个维度,系统梳理ABR刺激声参数优化的研究进展,分析现有技术的优势与局限,并探讨未来发展方向。
同时,建立标准化的 ABR 正常/异常反应识别方法也是该领域的另一项关键技术。通过制定明确的波形识别标准和量化评估指标,可确保检测结果的客观性和可重复性,这对临床诊断的准确性具有决定性意义。本文的核心内容即是对这两项关键技术进行深入探讨。

\section{刺激声类型的发展演进}
ABR刺激声类型的演进经历了从简单到复杂、从宽频到特异性的发展过程。Click声作为最早的ABR刺激声,这种瞬态声刺激具有频谱范围宽(主要集中在1-4kHz)、操作简便等特点,使其成为临床筛查的首选方法。研究表明,Click声诱发的ABR波形清晰稳定,波I-V分化良好,特别适合快速评估听觉通路的完整性。关于Click刺激的研究也一直在持续\cite{rocha2024click,lee2021paired,talge2018click,frontiers2024apd}。然而,随着临床需求的不断提高,Click声的局限性也逐渐显现:其宽频特性无法提供频率特异性信息,且在高强度刺激时容易产生刺激伪迹,这些缺陷促使研究者寻求更优化的刺激声类型。


接着,Tone-Burst\cite{orsini2004notched,rasetshwane2013latency,diao2011filter,search2008toneassr}刺激声的出现标志着ABR技术向频率特异性评估迈出了重要一步。Picton等学者\cite{picton1981auditory}系统研究了不同频率Tone-Burst刺激的ABR特征,发现采用Blackman或Hanning窗函数可以有效减少频谱旁瓣,提高频率特异性。Stapells团队\cite{stapells2000meta}的后续研究证实,优化后的Tone-Burst刺激可以在250-8000Hz范围内获得可靠的频率特异性反应,为临床听力评估提供了重要工具。然而,Tone-Burst刺激也存在明显不足:低频段(特别是250Hz和500Hz)的ABR波形振幅较低,信噪比差;同时,完整的频率特异性评估需要测试多个频点,耗时较长,这在婴幼儿测试中尤为突出。


Dau等学者\cite{dau2000optimized}提出的Chirp刺激声\cite{ceylan2025_nbcechirp,chirp_bic2023,derived2022_tailoredchirp}代表了ABR技术的新突破。这种基于基底膜行波延迟理论设计的刺激声,通过调整不同频率成分的相位关系,使声能量在耳蜗内同步到达最佳位置,从而显著增强神经同步放电。临床研究表明,与传统的Click声相比,Chirp刺激声的信噪比提升显著\cite{fobel2004optimal}。特别是在婴幼儿听力评估中,Chirp ABR的检出率明显高于传统方法\cite{vanmaanen2013chirp,gorga2017chirp}。

CE-Chirp系列刺激声通过延迟不同频率成分的发射时间(频散补偿机制),显著提升了ABR波V振幅。Cho等人(2015)\cite{cho2015auditory}的对照实验表明,在感音神经性耳聋患者中,CE-Chirp比Click刺激的波V检出率提高23\%(p<0.01)[2]。这一发现尤其适用于新生儿筛查,Cobb与Stuart(2016)\cite{cobb2016neonate}进一步证实,CE-Chirp的倍频程分频刺激可使早产儿ABR阈值信噪比提升4.7 dB。
然而,窄带刺激声的应用存在频率特异性与检测耗时的矛盾。Talaat团队(2020)\cite{talaat2020hearing}对比了窄带Chirp与Tone-Burst在儿童中的表现,发现虽然500Hz窄带Chirp的阈值检测准确性更高(92\% vs 85\%),但完成全频段测试需要额外15分钟。这一结果提示,刺激范式的选择需权衡临床效率与诊断精度。值得注意的是,Dávalos-González最新研究(2025)\cite{davalos2025auditory}指出,当CE-Chirp刺激强度低于30dB nHL时,其与Click的阈值差异不再显著(p=0.34),这为阈值附近刺激声选择提供了新证据。
同地,Chirp声的推广也面临挑战:一方面,其效果依赖于精确的相位控制,对设备要求较高;另一方面,不同厂商开发的Chirp声参数差异较大,缺乏统一标准。


值得注意的是,近年来还出现了一些新型刺激声的研究探索。例如,Notched Noise掩蔽Click声被用于评估特定高频区域(如8kHz)的听力功能\cite{nakamura2021notched,li2020notched};Speech-ABR采用的da刺激声则致力于研究听觉中枢对言语信号的编码特性。这些新技术虽然尚未成为临床常规,但为ABR的未来发展提供了新的可能性。

\section{刺激强度的优化研究}
刺激强度的选择直接影响ABR检测的敏感性和特异性。传统ABR检测多采用固定强度刺激(如65-80 dB nHL),这种方法虽然操作简便,但忽视了不同个体的听觉敏感性差异。研究表明,固定强度刺激在婴幼儿和老年人群中可能产生较大误差\cite{johnson2010variability}。
针对这一问题,研究者提出了多种强度优化方案。Hall\cite{hall2007new}通过系统研究建议,刺激强度应根据检测目的动态调整:筛查可采用较高强度(70-80 dB nHL),而阈值评估则需要从高强度开始,以5-10dB步进递减至反应消失。这种强度递减法显著提高了阈值评估的精确度,但测试时间相应延长。
近年来,一些智能化的强度调节算法逐渐应用于临床。Zhou\cite{zhou2010adaptive,zhou2012adaptive}开发的适应性强度调节算法,可以根据实时波形质量自动调整刺激强度,在保证信噪比的前提下尽量减少刺激强度。这种方法在婴幼儿测试中表现出色,然而,这类算法在极重度听力损失患者中的应用效果仍有待验证。
另一个重要进展是刺激强度校准方法的改进。传统nHL(Normal Hearing Level)校准基于正常听力年轻人的主观阈值,可能不适用于特殊人群。为此,研究者提出了基于个体主观听阈的SL(Sensation Level)校准方法\cite{leon1999calibration,rance2002method}。临床数据显示,采用dB SL校准可以显著提高婴幼儿和老年人群的测试准确性,但这种方法需要预先获取行为听阈,在无法配合行为测听的患者中应用受限。

\section{刺激速率的影响与优化}
刺激速率是影响ABR检测效率的关键参数。早期研究多采用较低的刺激速率(10-20次/秒),以获得清晰的单次波形。然而,这种低速率导致测试时间过长,在临床应用中受到限制。
随着信号处理技术的发展,高速率刺激(30-50次/秒)逐渐成为可能。研究表明,在保证足够叠加次数的前提下,适当提高刺激速率可以显著缩短测试时间,而波形形态和潜伏期保持相对稳定\cite{choi2013clinical}。这一发现在新生儿听力筛查中尤为重要,使得大规模筛查成为可能。但高速率刺激也存在局限性:随着刺激速率提高,波V振幅会逐渐降低,潜伏期轻微延长\cite{burkard2007auditory}。
最后,多参数协同优化研究相对缺乏。刺激声类型、强度和速率之间存在复杂的交互作用,但目前的研究多关注单一参数优化,缺乏系统性考量。 本研究主要是针对这三方面进行大量混合测试,实验表明在宽带Chirp刺激在低强度(40 dB SL)下显著提高了波V振幅,窄带Chirp在4000Hz刺激下表现优异,在75 dB SL和40 dB SL下分别对波V振幅也有显著影响。

\section{传统自动检测方法}
随着信号处理和机器学习技术的发展,涌现出多种高效且鲁棒的方法。当前主流的四类自动检测技术包括时域分析、频域分析、统计检验、模板匹配、机器学习以及多方法融合。时域分析方法基于波形的振幅和潜伏期特征实现初步识别(Wang et al., 2021)\cite{wang2021time};频域分析通过功率谱和时频变换揭示信号的频率成分(Chen et al., 2022)\cite{chen2022frequency};统计检验方法如Hotelling's T²和改进的F检验通过多变量统计模型判断信号显著性(Liu et al., 2020)\cite{liu2020statistical};模板匹配则利用动态时间规整等算法自动匹配标准波形模板以定位关键波峰(Zhao et al., 2023)\cite{zhao2023template};

\section{机器学习方法}
在机器学习兴起之前,ABR 信号的处理主要依赖专家手动设计的特征参数。Arnold(1985)的里程碑式研究首次提出了基于波幅–潜伏期双阈值的客观检测算法,在假阳性率上(8.3\%)已接近资深临床医生的水平(6.1\%),为后续的特征工程方法奠定了基础。此后,Acır 与 Özdamar(2006)\cite{acir2006automatic}在特征工程方面取得突破,他们采用支持向量机(SVM)对时域的峰谷特征进行选择,实现了高达 89.2\% 的阈值检测准确率,但仍需依赖人工标注的峰识别步骤,这成为制约自动化的关键瓶颈。

随着时频分析工具的引入,小波变换部分缓解了这一问题。Dobrowolski 等人(2016)\cite{dobrowolski2016classification}基于 Daubechies 小波分解构建了多分辨率分析框架,即使在未明确标注波形成分的情况下,仍能达到 84\% 的分类准确率。与此同时,为解决噪声环境下的稳定性问题,Berninger(2014)\cite{berninger2014analysis}提出基于交叉相关的分析方法,通过计算交替采集的子平均波形间的相关系数(如 r > 0.8 视为有效响应),在 40 dB 背景噪声下仍能保持 92\% 的检测率[9]。Wang 等人(2021)\cite{wang2021real}则进一步将该方法实时化,开发出一款移动端 APP,实现“采集–分析”延迟低于 50 ms,为床旁检测提供了可行的技术路径。

整体来看,传统机器学习方法在 ABR 自动分析领域的发展呈现出阶段性演进的特点,从早期的特征工程优化逐步转向分类器性能的提升。以支持向量机(SVM)为代表的早期机器学习模型,尤其具有代表性。Acır 等人(2011)\cite{Acir2013}在后续研究中进一步系统性地评估了 SVM 在 ABR 波形分类任务中的有效性,提出一套包含时域(如波峰振幅、潜伏期)、频域(如功率谱密度)和非线性特征(如样本熵)的综合性特征提取方案,并结合径向基核函数构建分类模型,其整体表现显著优于基于阈值规则的传统分析方法。

除了 SVM,集成学习方法也开始在 ABR 研究中崭露头角。McKearney(2019)比较了多种机器学习模型在 ABR 波形分类中的表现,包括决策树、随机森林及 SVM 等,指出决策树方法在保证一定准确率的同时,具备较好的可解释性,适合临床人工审阅和辅助决策。

然而,这一类方法仍然面临根本性局限。Ballachanda\cite{ballachanda1992adaptation}的研究早已指出,ABR 波形形态会随着刺激强度变化而发生非线性改变,这使得固定的特征提取规则难以适应动态变化的生理信号,影响模型的泛化能力与鲁棒性。模型性能高度依赖于人工提取的特征质量,这成为传统机器学习方法难以克服的核心障碍,也正是这一瓶颈,推动了深度学习方法的兴起,开启了 ABR 自动分析的新篇章。


\section{深度学习的应用}
深度学习技术的引入彻底重塑了 ABR 信号分析的范式。Yi 等人(2025)\cite{liu2025comparison}在一项多中心研究中系统比较了 ResNet、DenseNet、RNN 等多种模型架构,发现浅层卷积神经网络(如五层 ResNet-18)在有限数据条件下表现最优,AUC 可达 0.92,且推理延迟低于 2ms/次。这一结果颠覆了“深层网络更优”的传统认知,提示 ABR 信号具备短时程、局部性强的结构特征,更适合由浅层网络建模。该团队还指出,多中心数据有效增强了模型的泛化能力,尤其在低信噪比与复杂噪声环境下表现出优越的鲁棒性;但同时,设备差异与采样标准不一所带来的数据异质性,显著增加了模型训练难度与部署成本,限制其在资源受限场景下的普适推广。

为进一步提升模型的可解释性与临床可用性,研究者引入注意力机制等结构创新。Ji 等人(2024)\cite{ji2024abr}提出的 ABR-Attention 模型可通过三维注意力图(时间–通道–空间)可视化波 V 检测依据,临床验证显示其关注区域与人工判读一致率达 91\%。Liang 等人(2024)\cite{liang2024automatic}则采用 Transformer 架构,在完全不依赖手工特征的前提下,实现了 ABR 波形的端到端自动分割,其对波 I–V 的识别 F1-score 达 0.88,标志着深度学习在成分级检测层面实现了关键突破。

尽管如此,当前深度模型仍面临“数据饥渴”困境。Wimalarathna(2022)\cite{wimalarathna2022machine}综述指出,约 78\% 的研究样本量不足 500 例,这在处理如听神经病等罕见病变时导致模型泛化能力受限。Ma 等人(2023)\cite{ma2023auditory}尝试采用数据增强策略(如添加电极噪声、肌电伪迹)扩充训练集,但所生成的合成数据与真实临床数据在分类任务上的一致性仅为 79\%,显示出算法层面的优化空间正接近性能天花板。

更值得关注的是,现有大多数研究将声刺激参数(类型、强度)视作固定输入,忽略了其与 ABR 响应之间的耦合机制。Yi 团队首次提出“刺激–响应协同优化框架”,将刺激模式的自适应选择与模型训练融合一体,打破了传统“刺激固定、分析优化”的研究范式。研究发现,不同刺激条件下 ABR 波形的时频结构存在显著差异,例如 Click 刺激产生的波形在卷积神经网络中表现出更强的时域局部性。通过动态刺激适配算法,即使是在固定模型结构(如 HybridNet)下,也能显著提升对特定类型听力损失的检出能力。

综上所述,深度学习正以前所未有的方式推动 ABR 自动分析从“规则驱动”向“数据驱动”转型。但模型性能的进一步提升,将越来越依赖于数据策略、建模结构与刺激设计三者之间的协同优化。

\section{当前挑战与未来方向}
综合现有研究,ABR自动化分析仍存在三个关键性挑战:

\begin{enumerate}
    \item \textbf{近阈值检测可靠性问题}:如\cite{shaheen2024abrpresto}的研究指出,当刺激强度接近听阈水平(<20\,dB)时,即便最优算法的假阴性率仍高达27\%,这严重限制了其在早期听力损失诊断中的应用价值。
    
    \item \textbf{个体差异适应性不足}:\cite{aloufi2023sex}的系统综述表明,性别和激素水平会导致ABR潜伏期产生最大达0.3\,ms的差异,而现有模型大多未考虑此类生理性变异因素。
    
    \item \textbf{临床可解释性缺陷}:虽然 Mckearney\cite{mckearney2025automated}开发的自动波标记算法达到了专家级判读准确率,但其决策过程仍被视为"黑箱",这种不可解释性严重阻碍了临床医生的信任和采纳。
\end{enumerate}

\section{本章小结}
该小节的介绍主要包括三个部分,ABR 支撑工作,刺激声类型和强度的选择 方法、机器学习和深度学习的应用。ABR的快速诊断是一个相对较为复杂的任务,本章首先对 ABR 的一些支撑技术做了系统概述,梳理了ABR检测的核心技术框架,包括刺激参数优化与标准化波形分析两大支柱,为后续快速诊断奠定理论基础。
然后重点解析刺激声类型(Click/Tone-Burst/Chirp)的演进与选择策略,以及刺激强度和速率的优化方法,通过参数协同提升信号质量。接着对比传统机器学习(SVM、决策树)与深度学习(CNN、注意力机制)在ABR波形解析中的进展,指出"刺激-响应"协同优化是突破现有诊断效率瓶颈的关键路径。ABR快速诊断的复杂性要求将基础技术、参数优化与智能算法纳入统一框架进行研究,这正是本研究的核心创新点。